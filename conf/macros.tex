%%%%%%%%%%%%%%%%%%%%%%%%%%%%%%%%%%%%%%%%%%%%%%%%%%%%%%%%%%%%%%%%%%%%%
% En este fichero se incluyen las macros para trabajar cómodamente  %
%%%%%%%%%%%%%%%%%%%%%%%%%%%%%%%%%%%%%%%%%%%%%%%%%%%%%%%%%%%%%%%%%%%%%

% IIIIIIIIIIIIIIIIIIIIIIIIIIIIIIIIIIIIIIIIIIIIIIIIIIII
% II            DESCRIPCIÓN DE LAS MACROS           II
% IIIIIIIIIIIIIIIIIIIIIIIIIIIIIIIIIIIIIIIIIIIIIIIIIIII

% TODO macro incluir imagenes de manera flexible

% Incluir codigo c++ directamente de fuentes
% \codigoc{Opciones de listing}{ruta relativa fichero}

% Incluir table con entorno ltxtable desde carpeta tablas. Ejemplo achura=\textwidth
% \tabla{Nombre del fichero}{anchura}

% Entorno enunciado. Caja con título enunciado coloreada de gris. Ejemplo para nuevos entoros
% \begin{enunciado} \\\end{enunciado}



% IIIIIIIIIIIIIIIIIIIIIIIIIIIIIIIIIIIIIIIIIIIIIIIIIIII
% II           IMPLEMENTACIÓN MACROS                II
% IIIIIIIIIIIIIIIIIIIIIIIIIIIIIIIIIIIIIIIIIIIIIIIIIIII

% Incluir codigo c++ directamente de fuentes
\newcommand{\codigoc}[2]{
	\lstinputlisting[language=C++, #2]{#1}
}

% Incluir table con entorno ltxtable desde carpeta tablas. Ejemplo achura=\textwidth
\newcommand{\tabla}[2]{\LTXtable{#2}{./tablas/#1}} 

%%%%%%%%%%%%%%%%%%%%%%%%%%%%%%%%%%%%%%%%%%%%%%%%%%%%%%%%%%%%%%%%%%%%%%%%%%%%%%%%%%%%%%%%%%%%%%%%%%%%%%%%%%%%%%%%%%%%%%%%%%%%%%%%%%%%%%%%%%%%%%%%%%%%%%%%%%%%%%%%%%%%%%%%%%%%%%%%%%%%%%%%%%%%%%%%%%%%%%%%%%%%%%%%%%%%%%%%%%%%%%%%%%%%%%%%%%%%%%%%%%%%%%%%%%%%%%%%%%%%%%%%%%%%%%%%%%%%%%%%%%%%%%%%%%%%%%%%%%%%%%%%%%%%%%%%%%%%%%%%%%%%%%%%%%%%%%%%%%%%%%%%%%%%%%%%%%%%%%%%%%%%%%%%%%%%%%%%%%%%%%%%%%%%%%%%%%%%%%%%%%%%%%%%%%%%%%%%%%%%%%%%%%%%%%%%%%%%%%%%%%%%%%%%%%%%%%%%%%%%%%%%%%%%%%%%%%%%%%%%%%%%%%%%%%%%%%%%%%%%%%%%%%%%%%%%%%%%%%%%%%%%%%%%%%%%%%%%%%%%%%%%%%%%%%%%%%%%%%%%%%%%%%%%%%%%%%%%%%%%%%%%%%%%%%%%%%%%%%%%%%%%%%%%%%%%%%%%%%%%%%%%%
% Esta macro inserta una imagen centrada que esté en la carpeta imágenes con título.
% Pone como etiqueta para citar el fichero el nombre del fichero
%
%El formato es \imagen{Nombre del fichero en la carpeta}{Descripción/titulo de la imagen}{Anchura}
%
\newcommand{\imagen}[3]{
	\begin{center}
		\includegraphics[width=#3]{#1}
		\captionof{figure}{#2}
		\label{#1}
	\end{center}
}


% paquete y macro para insertarla imagen en un sitio predeterminado y exacto. Igual que comando imagen, pero se llamará imagenaqui

\newcommand{\imagenaqui}[3]{
	\begin{figure}[H]
		\begin{center}
			\includegraphics[width=#3]{#1}
			\caption{#2}
			\label{#1}
		\end{center}
	\end{figure}
}

%El formato es \imagen{Nombre del fichero en la carpeta}{Descripción/titulo de la imagen}{Anchura}
%
\newcommand{\imagenescala}[3]{
	\begin{figure}[H]
		\begin{center}
			\includegraphics[scale=#3]{#1}
			\captionof{figure}{#2}
			\label{#1}
		\end{center}
	\end{figure}
}

% IIIIIIIIIIIIIIIIIIIIIIIIIIIIIIIIIIIIIIIIIIIIIIIIIIII
% II           IMPLEMENTACIÓN ENTORNOS              II
% IIIIIIIIIIIIIIIIIIIIIIIIIIIIIIIIIIIIIIIIIIIIIIIIIIII

% Entorno enunciado. Caja con título enunciado coloreada de gris. Ejemplo para nuevos entoros
\newenvironment{enunciado}{% Entorno para los enunciados
    \ifx\salvarSangria\undefined
    \newlength{\salvarSangria}
    \fi
    \setlength{\salvarSangria}{\parindent}
    \begin{tcolorbox}[title=Enunciado,parbox=false]
    \setlength{\parindent}{\salvarSangria}
}{
	\end{tcolorbox}
	\setlength{\parindent}{\salvarSangria}
}
  
%\newenvironment{enunciado}
  %{Que hacer antes}
  %{Que hacer después}