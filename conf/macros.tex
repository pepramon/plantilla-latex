%%%%%%%%%%%%%%%%%%%%%%%%%%%%%%%%%%%%%%%%%%%%%%%%%%%%%%%%%%%%%%%%%%%%%
% En este fichero se incluyen las macros para trabajar cómodamente  %
%%%%%%%%%%%%%%%%%%%%%%%%%%%%%%%%%%%%%%%%%%%%%%%%%%%%%%%%%%%%%%%%%%%%%

% IIIIIIIIIIIIIIIIIIIIIIIIIIIIIIIIIIIIIIIIIIIIIIIIIIII
% II            DESCRIPCIÓN DE LAS MACROS           II
% IIIIIIIIIIIIIIIIIIIIIIIIIIIIIIIIIIIIIIIIIIIIIIIIIIII


% IIIIIIIIIIIIIIIIIIIIIIIIIIIIIIIIIIIIIIIIIIIIIIIIIIII
% II                  IMPLEMENTACIÓN                II
% IIIIIIIIIIIIIIIIIIIIIIIIIIIIIIIIIIIIIIIIIIIIIIIIIIII

% Insertar codigo en c++ desde un fichero. El segundo parámetro son datos adicionales
% \codigoc{ficehro con codigo en c++}{datos adicionales}
% Ejemplo
% \codigoc{./programa/datos.cpp}{}
\newcommand{\codigoc}[2]{
	\lstinputlisting[language=C++, #2]{#1}
}

%%%%%%%%%%%%%%%%%%%%%%%%%%%%%%%%%%%%%%%%%%%%%%%%%%%%%%%%%%%%%%%%%%%%%%%%%%%%%%%%%%%%%%%%%%%%%%%%%%%%%%%%%%%%%%%%%%%%%%%%%%%%%%%%%%%%%%%%%%%%%%%%%%%%%%%%%%%%%%%%%%%%%%%%%%%%%%%%%%%%%%%%%%%%%%%%%%%%%%%%%%%%%%%%%%%%%%%%%%%%%%%%%%%%%%%%%%%%%%%%%%%%%%%%%%%%%%%%%%%%%%%%%%%%%%%%%%%%%%%%%%%%%%%%%%%%%%%%%%%%%%%%%%%%%%%%%%%%%%%%%%%%%%%%%%%%%%%%%%%%%%%%%%%%%%%%%%%%%%%%%%%%%%%%%%%%%%%%%%%%%%%%%%%%%%%%%%%%%%%%%%%%%%%%%%%%%%%%%%%%%%%%%%%%%%%%%%%%%%%%%%%%%%%%%%%%%%%%%%%%%%%%%%%%%%%%%%%%%%%%%%%%%%%%%%%%%%%%%%%%%%%%%%%%%%%%%%%%%%%%%%%%%%%%%%%%%%%%%%%%%%%%%%%%%%%%%%%%%%%%%%%%%%%%%%%%%%%%%%%%%%%%%%%%%%%%%%%%%%%%%%%%%%%%%%%%%%%%%%%%%%%%%
% Nuevo comando para insertar una tabla. En este caso estarán dentro de la carpeta tablas, ver la plantilla de la carpeta
% Se utiliza mediante
%
% \tabla{Nombre del fichero}{anchura}
%
% donde nombre del fichero es el que está en la carpeta tablas
% y la anchura es el tamaño que se quiere de tabla. Utilizar \textwidth si se quiere la anchura del texto.
\newcommand{\tabla}[2]{\LTXtable{#2}{./tablas/#1}} 

%%%%%%%%%%%%%%%%%%%%%%%%%%%%%%%%%%%%%%%%%%%%%%%%%%%%%%%%%%%%%%%%%%%%%%%%%%%%%%%%%%%%%%%%%%%%%%%%%%%%%%%%%%%%%%%%%%%%%%%%%%%%%%%%%%%%%%%%%%%%%%%%%%%%%%%%%%%%%%%%%%%%%%%%%%%%%%%%%%%%%%%%%%%%%%%%%%%%%%%%%%%%%%%%%%%%%%%%%%%%%%%%%%%%%%%%%%%%%%%%%%%%%%%%%%%%%%%%%%%%%%%%%%%%%%%%%%%%%%%%%%%%%%%%%%%%%%%%%%%%%%%%%%%%%%%%%%%%%%%%%%%%%%%%%%%%%%%%%%%%%%%%%%%%%%%%%%%%%%%%%%%%%%%%%%%%%%%%%%%%%%%%%%%%%%%%%%%%%%%%%%%%%%%%%%%%%%%%%%%%%%%%%%%%%%%%%%%%%%%%%%%%%%%%%%%%%%%%%%%%%%%%%%%%%%%%%%%%%%%%%%%%%%%%%%%%%%%%%%%%%%%%%%%%%%%%%%%%%%%%%%%%%%%%%%%%%%%%%%%%%%%%%%%%%%%%%%%%%%%%%%%%%%%%%%%%%%%%%%%%%%%%%%%%%%%%%%%%%%%%%%%%%%%%%%%%%%%%%%%%%%%%%
% Esta macro inserta una imagen centrada que esté en la carpeta imágenes con título.
% Pone como etiqueta para citar el fichero el nombre del fichero
%
%El formato es \imagen{Nombre del fichero en la carpeta}{Descripción/titulo de la imagen}{Anchura}
%
\newcommand{\imagen}[3]{
	\begin{center}
		\includegraphics[width=#3]{#1}
		\captionof{figure}{#2}
		\label{#1}
	\end{center}
}


% paquete y macro para insertarla imagen en un sitio predeterminado y exacto. Igual que comando imagen, pero se llamará imagenaqui

\newcommand{\imagenaqui}[3]{
	\begin{figure}[H]
		\begin{center}
			\includegraphics[width=#3]{#1}
			\caption{#2}
			\label{#1}
		\end{center}
	\end{figure}
}

%El formato es \imagen{Nombre del fichero en la carpeta}{Descripción/titulo de la imagen}{Anchura}
%
\newcommand{\imagenescala}[3]{
	\begin{figure}[H]
		\begin{center}
			\includegraphics[scale=#3]{#1}
			\captionof{figure}{#2}
			\label{#1}
		\end{center}
	\end{figure}
}

%%%%%%%%%%%%%%%%%%%%%%%%%%%%%%%%%%%%%%%%%%%%%%%%%%%%%%
%%%%%%%%%%%%%%%%%%%%%%%%%%%%%%%%%%%%%%%%%%%%%%%%%%%%%%
%%%%%%%%%%%%%%%%%%%%%%%%%%%%%%%%%%%%%%%%%%%%%%%%%%%%%%
%%%                                                %%%
%%%   ______       _                               %%%
%%%  |  ____|     | |                              %%%
%%%  | |__   _ __ | |_ ___  _ __ _ __   ___  ___   %%%
%%%  |  __| | '_ \| __/ _ \| '__| '_ \ / _ \/ __|  %%%
%%%  | |____| | | | || (_) | |  | | | | (_) \__ \  %%%
%%%  |______|_| |_|\__\___/|_|  |_| |_|\___/|___/  %%%
%%%                                                %%%
%%%                                                %%%
%%%%%%%%%%%%%%%%%%%%%%%%%%%%%%%%%%%%%%%%%%%%%%%%%%%%%%
%%%%%%%%%%%%%%%%%%%%%%%%%%%%%%%%%%%%%%%%%%%%%%%%%%%%%%
%%%%%%%%%%%%%%%%%%%%%%%%%%%%%%%%%%%%%%%%%%%%%%%%%%%%%%


\newenvironment{enunciado}{% Entorno para los enunciados
    \ifx\salvarSangria\undefined
    \newlength{\salvarSangria}
    \fi
    \setlength{\salvarSangria}{\parindent}
    \begin{tcolorbox}[title=Enunciado,parbox=false]
    \setlength{\parindent}{\salvarSangria}
  }
	{
	\end{tcolorbox}
	\setlength{\parindent}{\salvarSangria}
  }
  
%\newenvironment{enunciado}
  %{Que hacer antes}
  %{Que hacer después}