%%%%%%%%%%%%%%%%%%%%%%%%%%%%%%%%%%%%%%%%%%%%%%%%%%%%%%%%%%%%%%%%%%%%%
% En este fichero se incluyen las macros para trabajar cómodamente  %
%%%%%%%%%%%%%%%%%%%%%%%%%%%%%%%%%%%%%%%%%%%%%%%%%%%%%%%%%%%%%%%%%%%%%

% IIIIIIIIIIIIIIIIIIIIIIIIIIIIIIIIIIIIIIIIIIIIIIIIIIII
% II            DESCRIPCIÓN DE LAS MACROS           II
% IIIIIIIIIIIIIIIIIIIIIIIIIIIIIIIIIIIIIIIIIIIIIIIIIIII

% INCLUIR IMAGENES
% \imagen[opt]{nombre fichero}{titulo de la imagen}
% Opt:
%   ancho= ancho de la imagen (por defecto escala)
%   escala = escala de la imagen (defecto 1)
%   etiqueta = \label por defecto, nombre fichero
%   aquí --> opción H de figura. Defecto fase (htb)
%   carpeta = nombre carpeta donde están imagenes
%     ejemplo "carpeta=img/", defecto ./
% Configuración general
\def\ImagenDefecto{aqui, escala=0.5}

% INCLUIR CODIGO C++ DESDE FUENTES
% \codigoc{Opciones de listing}{ruta relativa fichero}

% INCLUIR TABLAS LARGAS (dentro carpeta tablas)
% \tabla{Nombre del fichero}{anchura}

% ENTORNO ENUNCIADO (titulo: enunciado. Caja gris)
% \begin{enunciado} \\\end{enunciado}


% IIIIIIIIIIIIIIIIIIIIIIIIIIIIIIIIIIIIIIIIIIIIIIIIIIII
% II           IMPLEMENTACIÓN MACROS                II
% IIIIIIIIIIIIIIIIIIIIIIIIIIIIIIIIIIIIIIIIIIIIIIIIIIII

%%%%%%%%%%%%%%%%%%%%%%%%%%%%%%%%%%%%%%%%%%%%%%%%%%%%%%%%%%%%%
%%%%%%%%%%%%%%%%%%%%%%%%%%%%%%%%%%%%%%%%%%%%%%%%%%%%%%%%%%%%%
% ENTORNO ENUNCIADO (titulo: enunciado. Caja gris)
\newcommand{\codigoc}[2]{
	\lstinputlisting[language=C++, #2]{#1}
}

%%%%%%%%%%%%%%%%%%%%%%%%%%%%%%%%%%%%%%%%%%%%%%%%%%%%%%%%%%%%%
%%%%%%%%%%%%%%%%%%%%%%%%%%%%%%%%%%%%%%%%%%%%%%%%%%%%%%%%%%%%%
% INCLUIR TABLAS LARGAS (dentro carpeta tablas)
\newcommand{\tabla}[2]{\LTXtable{#2}{./tablas/#1}} 

%%%%%%%%%%%%%%%%%%%%%%%%%%%%%%%%%%%%%%%%%%%%%%%%%%%%%%%%%%%%%
%%%%%%%%%%%%%%%%%%%%%%%%%%%%%%%%%%%%%%%%%%%%%%%%%%%%%%%%%%%%%
% INCLUIR IMAGENES
%%%%%%%%%%%%%% Creación de las claves
\makeatletter
\define@key{Imagen}{carpeta}{\def\Imagen@carpeta{#1}}
\define@key{Imagen}{ancho}{\def\Imagen@ancho{#1}}
\define@key{Imagen}{escala}{\def\Imagen@escala{#1}}
\define@key{Imagen}{aqui}[true]{\def\Imagen@aqui{#1}}
\define@key{Imagen}{etiqueta}{\def\Imagen@etiqueta{#1}}
%%%%%%%%%%%%%% Valores por defecto
\setkeys{Imagen}{\ImagenDefecto}
%%%%%%%%%%%%%% Definición de la Macro
\newcommand{\imagen}[3][]{
 %se empieza un grupo para que no guarde
 \begingroup
  \setkeys{Imagen}{#1}
  \ifdef\Imagen@carpeta%
    {\def\Imagen@fichero{{\Imagen@carpeta#2}}}%
    {\def\Imagen@fichero{#2}}
  \ifdef\Imagen@aqui%
    {\begin{figure}[H]}%
    {\begin{figure}[htb]}
      \begin{center}
      	\ifdef\Imagen@anchura%
      	  {\includegraphics[width=\Imagen@ancho]{\Imagen@fichero}}%
      	  {\ifdef\Imagen@escala%
      	    {\includegraphics[scale=\Imagen@escala]{\Imagen@fichero}}
	        {\includegraphics{\Imagen@fichero}}
	      }
        \captionof{figure}{#3}
        \ifdef\Imagen@etiqueta%
          {\label{\Imagen@etiqueta}}%
          {\label{#2}}
      \end{center}
    \end{figure}
 \endgroup
}
\makeatother

% IIIIIIIIIIIIIIIIIIIIIIIIIIIIIIIIIIIIIIIIIIIIIIIIIIII
% II           IMPLEMENTACIÓN ENTORNOS              II
% IIIIIIIIIIIIIIIIIIIIIIIIIIIIIIIIIIIIIIIIIIIIIIIIIIII

%%%%%%%%%%%%%%%%%%%%%%%%%%%%%%%%%%%%%%%%%%%%%%%%%%%%%%%%%%%%%
%%%%%%%%%%%%%%%%%%%%%%%%%%%%%%%%%%%%%%%%%%%%%%%%%%%%%%%%%%%%%
% ENTORNO ENUNCIADO (titulo: enunciado. Caja gris)
\newenvironment{enunciado}{% Entorno para los enunciados
    \ifx\salvarSangria\undefined
    \newlength{\salvarSangria}
    \fi
    \setlength{\salvarSangria}{\parindent}
    \begin{tcolorbox}[title=Enunciado,parbox=false]
    \setlength{\parindent}{\salvarSangria}
}{
	\end{tcolorbox}
	\setlength{\parindent}{\salvarSangria}
}
  
%\newenvironment{enunciado}
  %{Que hacer antes}
  %{Que hacer después}