%%%%%%%%%%%%%%%%%%%%%%%%%%%%%%%%%%%%%%%%%%%%%%%%%%%%%%%%%%%%%%%%%%%%%
% En este fichero se incluyen las macros para trabajar cómodamente  %
%%%%%%%%%%%%%%%%%%%%%%%%%%%%%%%%%%%%%%%%%%%%%%%%%%%%%%%%%%%%%%%%%%%%%

% IIIIIIIIIIIIIIIIIIIIIIIIIIIIIIIIIIIIIIIIIIIIIIIIIIII
% II            DESCRIPCIÓN DE LAS MACROS           II
% IIIIIIIIIIIIIIIIIIIIIIIIIIIIIIIIIIIIIIIIIIIIIIIIIIII

% INCLUIR IMAGENES
% \imagen[opt]{nombre fichero}{titulo de la imagen}
% genera etiqueta con nombre del fichero
% ver documentación del paquete easyfig
\newcommand{\imagen}[3][]{\Figure[placement=H,caption={#3},label=#2,#1]{#2}}

% INCLUIR CODIGO C++ DESDE FUENTES
% \codigoc{Opciones de listing}{ruta relativa fichero}
\newcommand{\codigoc}[2]{\lstinputlisting[language=C++, #2]{#1}}

% INCLUIR TABLAS LARGAS (dentro carpeta tablas)
% \tabla{Nombre del fichero}{anchura}
\newcommand{\tabla}[2]{\LTXtable{#2}{./tablas/#1}} 

% ENTORNO ENUNCIADO (titulo: enunciado. Caja gris)
% \begin{enunciado} \\\end{enunciado}
\newenvironment{enunciado}{% Entorno para los enunciados
    \ifx\salvarSangria\undefined
    \newlength{\salvarSangria}
    \fi
    \setlength{\salvarSangria}{\parindent}
    \begin{tcolorbox}[title=Enunciado,parbox=false]
    \setlength{\parindent}{\salvarSangria}
}{
	\end{tcolorbox}
	\setlength{\parindent}{\salvarSangria}
}

% GUARDADO PARA RECORDAR  
%\newenvironment{enunciado}
  %{Que hacer antes}
  %{Que hacer después}