%%%%%%%%%%%%%%%%%%%%%%%%%%%%%%%%%%%%%%%%%%%%%%%%%%%%%%%%%%%%%%%%%%%%%
% En este fichero se importan los paquetes necesarios para que      %
% funcione correctamente la plantilla y se configuran los paquetes  %
%%%%%%%%%%%%%%%%%%%%%%%%%%%%%%%%%%%%%%%%%%%%%%%%%%%%%%%%%%%%%%%%%%%%%

% IIIIIIIIIIIIIIIIIIIIIIIIIIIIIIIIIIIIIIIIIIIIIIIIIIII
% II                MATEMATICAS                     II
% IIIIIIIIIIIIIIIIIIIIIIIIIIIIIIIIIIIIIIIIIIIIIIIIIIII

\usepackage{amsmath}
\usepackage{bm}
\usepackage{amssymb}
\usepackage{cancel} %Paquete para poder tachar en modo matemático
\usepackage{mathtools}%Más herramientas para ecuaciones, por ejemplo el entorno flalign que permite poner las ecuaciones a la izquierda
\usepackage{fix-cm} % \texorpdfstring{matematicas}{substitucion}
\numberwithin{equation}{section} %reiniciar contador eq en seccion
\renewcommand\theequation{\thesection.\arabic{equation}} %nombre equacion

% IIIIIIIIIIIIIIIIIIIIIIIIIIIIIIIIIIIIIIIIIIIIIIIIIIII
% II                  GENERALES                     II
% IIIIIIIIIIIIIIIIIIIIIIIIIIIIIIIIIIIIIIIIIIIIIIIIIIII

\usepackage{import} % Importar fichero
\usepackage[spanish, es-tabla]{babel} %Nombres en español y aceptar acentos
  \spanishlcroman %Para que la numeración romana pueda salir en minusculas
\usepackage[utf8]{inputenc} % Codificación entrada
\usepackage[T1]{fontenc} %para incrustar fuente en PDF
\usepackage{verbatim}  %Insertar texto literal
\usepackage{float} %Mejora de elementos flotantes, insertar antes que hyperref
\usepackage{hyperref} %Links dentro del pdf
\usepackage{calc} %Habilitar cálculos
\usepackage{pdfpages} %Incluir pdf en documento
  \pdfsuppresswarningpagegroup=1 %Evitar errores inclusión 2 pdf en página
\usepackage{graphicx} %Incluir imagenes en documento
\usepackage[small]{caption} %Incluir títulos libremente
\usepackage{framed} %Para realizar recuadros
\usepackage{eurosym} %Simbolo del euro \euro{}
  \DeclareUnicodeCharacter{20AC}{\euro{}} % simbolo euro con caracter
\usepackage{easyfig} % para insertar facilmente imagenes
\usepackage{csquotes} %para evitar error con babel y biblatex

% IIIIIIIIIIIIIIIIIIIIIIIIIIIIIIIIIIIIIIIIIIIIIIIIIIII
% II                APARIENCIA                      II
% IIIIIIIIIIIIIIIIIIIIIIIIIIIIIIIIIIIIIIIIIIIIIIIIIIII

\usepackage[style=numeric,backend=biber]{biblatex} %Bibliografia para latex
\renewcommand{\baselinestretch}{1.3}\normalsize %interlineado
\usepackage{enumitem} %Espacios en listas itemize
  \setitemize{itemsep=-3pt,topsep=5pt,partopsep=0pt, parsep=5pt}
  \setenumerate{itemsep=-3pt,topsep=5pt,partopsep=0pt, parsep=5pt}
  \setdescription{itemsep=-3pt,topsep=5pt,partopsep=0pt, parsep=5pt}
\usepackage{indentfirst} % Sangrado en cada parrafo 
\setlength\parskip{0.3\baselineskip} %Separación entre párrafos
\hyphenrules{nohyphenation} %No partir palabras
\raggedbottom %No intentar distribuir el texto en vertical
\usepackage[scaled]{helvet} %Tipo de letra helvetica
  \renewcommand{\familydefault}{\sfdefault} %setear letras sansSerif
\usepackage[helvet]{sfmath} %Helvetica en equaciones
  \everymath={\sf} %setear en todos los entornos mat SS font
\usepackage[section]{placeins} %Forzar limpieza flotante por sección
\setcounter{tocdepth}{2} %Profundidad del índice (0 capítulo)
\setcounter{secnumdepth}{3} %Profundidad numeración (0 capítulo)
%\numberwithin{figure}{section} %Reiniciar contador figuras con seccion
%\renewcommand\thefigure{\thesection.\arabic{figure}} %nombre figura

% IIIIIIIIIIIIIIIIIIIIIIIIIIIIIIIIIIIIIIIIIIIIIIIIIIII
% II                FORMATO PÁGINA                  II
% IIIIIIIIIIIIIIIIIIIIIIIIIIIIIIIIIIIIIIIIIIIIIIIIIIII
\usepackage[a4paper, 			%Papel A4
  left=3.18cm, 					%Margen izquierdo
  right=2.54cm, 				%Margen derecho
  top=1.25cm, 					%Margen superior
  bottom=2.54cm, 				%Margen inferior
  includeheadfoot]{geometry}	%Incluir pie y cabecero en tamaño pag
\usepackage{fancyhdr} %definir pies y cabeza de pagina
  \fancypagestyle{normal}{ %Se define el estilo normal
    \fancyhf{} %Se limpia cualquier formato
    \fancyfoot[C,C]{- \thepage -} % Página centrada en todas la páginas
    %\fancyhead[C,C]{\thechapter.-\leftmark} %nº y nombre capitulo en cabecera
    \renewcommand{\headrulewidth}{0pt} %tamaño linea separación cabeza
    \renewcommand{\footrulewidth}{0.4pt} %tamaño linea separación pie
  }
  \fancypagestyle{solo-pie}{ %Se define el estilo normal
    \fancyhf{} %Se limpia cualquier formato
    \fancyfoot[C,C]{- \thepage -} % Página centrada en todas la páginas
    \renewcommand{\headrulewidth}{0pt} %tamaño linea separación cabeza
    \renewcommand{\footrulewidth}{0.4pt} %tamaño linea separación pie
  }
  \fancypagestyle{plain}{\pagestyle{solo-pie}} %Estilo plain en titulos capitulo e indice, se asigna el estilo solo-pie
  \pagestyle{normal} %Estilo general del documento
  \renewcommand{\chaptermark}[1]{\markboth{#1}{}} %Borrar "Capítulo" de leftmark

% IIIIIIIIIIIIIIIIIIIIIIIIIIIIIIIIIIIIIIIIIIIIIIIIIIII
% II                FORMATO TITULOS                 II
% IIIIIIIIIIIIIIIIIIIIIIIIIIIIIIIIIIIIIIIIIIIIIIIIIIII
\usepackage{titlesec}
  \titleformat{\chapter}[hang] %Formato capitulo tipo hang
    {\bfseries\fontsize{20}{20}\selectfont} %Formato texto
    {\thechapter.} %Etiqueta
    {3mm} %Separación entre etiqueta y nombre
    {} %comandos antes de el cuerpo del titulo
    [] %comandos después de el cuerpo del titulo
  \titlespacing{\chapter}
    {0cm} %Separación izquierda
    {1cm} %Separación antes del titulo
    {1cm} %Separación después del titulo
    [0cm] %Separación izquiera
  \titleformat{\section}[hang] %Formato capitulo tipo hang
    {\bfseries\fontsize{18}{18}\selectfont} %Formato texto
    {\thesection.} %Etiqueta
    {3mm} %Separación entre etiqueta y nombre
    {} %comandos antes de el cuerpo del titulo
    [] %comandos después de el cuerpo del titulo
  \titlespacing{\section}
    {0cm} %Separación izquierda
    {0.5cm} %Separación antes del titulo
    {0.5cm} %Separación después del titulo
    [0cm] %Separación izquiera
  \titleformat{\subsection}[hang] %Formato capitulo tipo hang
    {\bfseries\fontsize{16}{16}\selectfont} %Formato texto
    {\thesubsection.} %Etiqueta
    {3mm} %Separación entre etiqueta y nombre
    {} %comandos antes de el cuerpo del titulo
    [] %comandos después de el cuerpo del titulo
  \titlespacing{\subsection}
    {0cm} %Separación izquierda
    {0cm} %Separación antes del titulo
    {0cm} %Separación después del titulo
    [0cm] %Separación izquiera
  \titleformat{\subsubsection}[hang] %Formato capitulo tipo hang
    {\bfseries\itshape\fontsize{14}{14}\selectfont} %Formato texto
    {\thesubsubsection.} %Etiqueta
    {3mm} %Separación entre etiqueta y nombre
    {} %comandos antes de el cuerpo del titulo
    [] %comandos después de el cuerpo del titulo
  \titlespacing{\subsubsection}
    {0cm} %Separación izquierda
    {0cm} %Separación antes del titulo
    {0cm} %Separación después del titulo
    [0cm] %Separación izquiera
\usepackage{etoolbox}%Hack por un error en la versión de titlesec
  \makeatletter
  \patchcmd{\ttlh@hang}{\parindent\z@}{\parindent\z@\leavevmode}{}{}
  \patchcmd{\ttlh@hang}{\noindent}{}{}{}
  \makeatother
  
% IIIIIIIIIIIIIIIIIIIIIIIIIIIIIIIIIIIIIIIIIIIIIIIIIIII
% II            OTROS PAQUETES UTILIZADOS           II
% IIIIIIIIIIIIIIIIIIIIIIIIIIIIIIIIIIIIIIIIIIIIIIIIIIII

\usepackage{tcolorbox} %Para hacer cajas de colores
%\usepackage[spanish]{nomencl} %Paquete para generar nomenclatura
%  \makenomenclature
\usepackage{longtable} %para tablas largas
\usepackage{ltxtable} %incuir una tabla desde un documento externo
\usepackage{listings} %Insertar codigo fuente directamente
\lstset{frame=Ltb,
  framerule=0pt,
  aboveskip=0.5cm,
  framextopmargin=3pt,
  framexbottommargin=3pt,
  framexleftmargin=0.4cm,
  framesep=0pt,
  rulesep=.4pt,
  %
  stringstyle=\ttfamily,
  showstringspaces = false,
  basicstyle=\small\ttfamily,
  keywordstyle=\bfseries,
  %
  numbers=left,
  numbersep=15pt,
  numberstyle=\tiny,
  numberfirstline = false,
  breaklines=true,
  escapeinside={*@}{@*},
  literate=
    {á}{{\'a}}1 {é}{{\'e}}1 {í}{{\'i}}1 {ó}{{\'o}}1 {ú}{{\'u}}1
    {Á}{{\'A}}1 {É}{{\'E}}1 {Í}{{\'I}}1 {Ó}{{\'O}}1 {Ú}{{\'U}}1
    {à}{{\`a}}1 {è}{{\`e}}1 {ì}{{\`i}}1 {ò}{{\`o}}1 {ù}{{\`u}}1
    {À}{{\`A}}1 {È}{{\'E}}1 {Ì}{{\`I}}1 {Ò}{{\`O}}1 {Ù}{{\`U}}1
    {ä}{{\"a}}1 {ë}{{\"e}}1 {ï}{{\"i}}1 {ö}{{\"o}}1 {ü}{{\"u}}1
    {Ä}{{\"A}}1 {Ë}{{\"E}}1 {Ï}{{\"I}}1 {Ö}{{\"O}}1 {Ü}{{\"U}}1
    {â}{{\^a}}1 {ê}{{\^e}}1 {î}{{\^i}}1 {ô}{{\^o}}1 {û}{{\^u}}1
    {Â}{{\^A}}1 {Ê}{{\^E}}1 {Î}{{\^I}}1 {Ô}{{\^O}}1 {Û}{{\^U}}1
    {œ}{{\oe}}1 {Œ}{{\OE}}1 {æ}{{\ae}}1 {Æ}{{\AE}}1 {ß}{{\ss}}1
    {ç}{{\c c}}1 {Ç}{{\c C}}1 {ø}{{\o}}1 {å}{{\r a}}1 {Å}{{\r A}}1
    {€}{{\EUR}}1 {£}{{\pounds}}1 {ñ}{{\~n}}1 {Ñ}{{\~N}}1 {·}{{*}}1 
    {–}{{-}}1 {_}{{\textunderscore}}1
}